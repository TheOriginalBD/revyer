\documentclass[a4paper,12pt]{article}

\usepackage{revy}
\usepackage[utf8]{inputenc}
\usepackage[T1]{fontenc}
\usepackage[danish]{babel}
\pdfoptionpdfminorversion 4

\revyname{Biorevy}
\revyyear{2012}
% HUSK AT OPDATERE VERSIONSNUMMER
% UNDLAD AT SKRIVE I TEMPLATE.TEX - KOPIÉR OG OMDØB I STEDET FOR
\version{2}
\eta{$X$ minutter}
\status{Placeholder}

\title{Biosfæren}
\author{Markus Drag, Biorevy}

\begin{document}
\maketitle

\begin{texxers}
	\texxer{Thomas BT}[thomas.bech.thomassen@gmail.com]
\end{texxers}

\begin{roles}
	\role{Stud}[Jennie B] Studerende fra Biosfære-udvalget
	\role{Insp}[Thomas BT] Bygnings-inspektør
	\role{AV}[AV] 
\end{roles}


\begin{props}
\prop{Lydmåler}[Hvem skaffer?] Lysende-øre dB-måler
\prop{Bygningsinspektør uniform} [Thomas BT] Skjorte?
\prop{Bord-ventilator}[Lykke P]
\prop{Clipboard}[Thomas BT] Clipboard m. Reglerne
\end{props}

\begin{sketch}

\scene Stud [glad] og insp. [alvorlig] kommer ind på scenen. Stud slår omkring sig med armene.

\says{Stud} Wow, biologis helt nye studenterrum, Biosfæren! Hvor er her flot!

\scene Stud står lidt og ser glad ud, Insp. går i adstadigt tempo hen og rømmer sig, prikker hende på skulderen.

\says{Insp} Ja, ja. Lad os nu lige slå koldt vand i blodet. Der er jo \act{Pause -- Vender en side på clipboardet} \emph{reglerne}.

\says{Stud} Ja, naturligvis. Vi er bare glade for, at vi får vores eget sted som biologerne har adgang til.

\says{Insp}[sagt hurtigt] Ja, men, at I ikke har adgang med jeres studiekort. Det skal først godkendes af dekanatet, refereres via prodekanen til jeres studieleder. Så skal det ind forbi centralkontoret. Og så tilbage til mig med en skriftlig godkendelse. Og jeg er sikkert syg den dag\ldots

\says{Stud}[forvirret] Ja. Naturligvis. Men det vigtigste er også vi har et sted at kunne samles i frokostpausen\ldots

\says{Insp} Men, husk nu at I absolut ikke må være her i frokostpausen. Kantinen kunne jo miste penge!

\says{Stud}[delvist nedslået] \ldots Men vi er også bare glade for at kunne ses, og have det sjovt efter timerne.

\scene Insp. trækker Stud hen til lydmåleren, der hænger til venstre på scenen.

\says{Insp} Ja, altså, I må ikke ha' det sjovt hernede. Grin og latter larmer ALT for meget . Alt over stille er absolut ikke acceptabelt. Tænk på laboratoriemusene!

\says{Stud} Ja, jo naturligvis \act{nedslået}. Men heldigvis har vi da nu et sted vi kan indrette til et rigtigt biolog-rum!

\scene Stud går hen og ruller den Biorevy 2012-plakat hun har haft med på scenen ud.

\says{Insp} Husk \act{trykker Stud hånd op, sp plakaten ruller sammen}, at man absolut ikke må hænge noget op på væggene  -- eller ændre det \emph{arkitekttegnede betonlook}!

\says{Stud}[meget nedslået] Okay, men i det mindste har vi da et rum\ldots eller\ldots

\says{AV}[Lyd] Lyden af noget mekanisk, der går i stykker 

\scene Bordventilator, der har været tændt hele tiden og stået i hjørnet på gulvet, stopper pludselig

\says{Insp} Hov! \act{går hen til ventilator}. Nej, det var dog uheldigt!

\says{Stud} Hvad er det?

\says{Insp} Ja, så røg ventilatoren altså igen. Der skal bestilles en ny del fra Schweiz. 

\says{Stud}[rystet] Hvor lang tid tager det? 

\says{Insp} Åhr, ikke så længe. \act{Pause} \ldots Nok bare to år.

\scene Lys ned.





\end{sketch}
\end{document}
