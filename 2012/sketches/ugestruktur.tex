\documentclass[a4paper,12pt]{article}

\usepackage{revy}
\usepackage[utf8]{inputenc}
\usepackage[T1]{fontenc}
\usepackage[danish]{babel}
\pdfoptionpdfminorversion 4

\revyname{Biorevy}
\revyyear{2012}
% HUSK AT OPDATERE VERSIONSNUMMER
% UNDLAD AT SKRIVE I TEMPLATE.TEX - KOPIÉR OG OMDØB I STEDET FOR
\version{4}
\eta{$X$ minutter}
\status{Færdig}

\title{Ugestruktur}
\author{Julie K. Sheard, Biorevy}

\begin{document}
\maketitle

\begin{texxers}
	\texxer{Thomas BT}[thomas.bech.thomassen@gmail.com]
\end{texxers}

\begin{roles}
	\role{HB}[Markus D] Henrik Busch
	\role{VO}[Thomas BT] Voice over
        \role{AV}[AV]
\end{roles}


\begin{props}
	\prop{Slideshow}[Julie KS] Ugestrukutur-slideshow'et
	\prop{Billeder}[Markus D] Fødselsdag + NBSP
\end{props}

\begin{sketch}


\scene Scenen er tom til at starte med.
\says{VO}[dramatisk stemme] Mine damer og herrer, vi har indkaldt til dette orienteringsmøde om den fremtidige undervsinings-struktur på SCIENCE. Vi beklager at møde ligger i juleferien, og at i fik det at vide i går. Byd velkommen jeres alle sammens prodekan: \act{Pause} \\ \emph{Henriiik Buuuuuuuuuuuuuuuusch}!

\scene Spot på HB

\says{HB} I sommer introducerede ledelsen til jer en idyllisk oase i det travle studiemiljø, nemlig Campusvognen. Desværre blev ideen ikke modtaget med den forventede entusiasme og ledelsen måtte erklære projektet for en fiasko, efter at have fundet CampusVognen totaltsmadret.

\says{HB}DEN SLAGS HÆRVÆRK VIL VI IKKE TOLERERE! 

\says{HB}Men frygt ej! Når nu I, de studerende, har givet udtryk for at I IKKE ønsker et bedre, moderne og indbydende studiemiljø, har ledelsen valgt at sætte studieaktiviteten i fokus. 

\says{HB}Tilbage i 2006 overgik det Naturvidenskabelige Fakultet fra Semester- til Blokstruktur. Nu seks år senere kan vi konstatere at studieaktiviteten er steget med 20 procent \ldots Det skyldes uden tvivl overgangen.

\says{AV} Slide med voldsomt øget studieaktivitet som følge af overgang til blokstruktur.

\says{HB} DERFOR vil SCIENCE, fra Nytår, overgå til den nye fantastiske UGESTRUKTUR! Ved at opdele blokken i 8 uger regner vi derfor med at se en stigning i studieaktivitet på hele 80 procent !

\says{HB}Det vil altså sige at I, kære studerende, kommer til at mærke en lille - men positiv - ændring.

\says{HB}I stedet for at bruge et HELT år på at tage førsteårs kurserne, vil I blot skulle have en uge med hvert kursus: Organismernes Diversitet, Matematik/Statistik, Populationsbiologi, Almen Kemi, Almen Økologi,og Sommerkurser! Pensum er selvfølgelig uændret!

\says{HB}I hver uge vil lørdag være læseferie, og søndag vil I skulle gå til eksamen. 
Og bliver det ikke bare dejligt at have overstået ALT det arbejde på blot een enkelt skaldet lille blok? 
\says{HB}Nu tænker I nok; Åh Henrik, du ved da også lige hvad vi studerende ønsker. Men hvornår skal vi så feste?
\act{Er først alvorlig og så selvtilfreds.}

\says{HB}Jeg ved godt at I unge mennesker har brug for jeres rusmidler. Jeg har set jer på Biobar, euforiske, dansende med galskaben lysende i blikket, tørstende efter særdeles uprofessionelt assistance fra jeres medstuderende. FØJ!!!

\act{Spytter og harker, så slipset svinger omkring.}
\says{HB}Men nuvel. Jeg er måske blevet sentimental på mine gamle dage. Derfor har jeg indlagt en VEJLEDNINGSDAG, i slutningen af hver blok hvor I kan få lidt ro. 
Bemærk at der på mit slide er en lille stjerne.
Hvis der er nogen der ikke allerede skulle være klar over det - så indikerer denne stjerne at det på næste vejledningsdag er min fødselsdag.

\says{HB}Jeg forventer derfor at se jer alle foran Green Lighthouse, hvor jeg vil modtage min hyldest. Jeg behøver ikke nævne at der naturligvis er mødepligt.
Derudover er de næste 100 vejledningsdage afsat til opførelse af Niels bohr Science Park. Husk Skovl.
\says{AV}[Slide]Billede af studerende, der tager spadestik i Uniparken



\end{sketch}
\end{document}
