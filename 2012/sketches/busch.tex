\documentclass[a4paper,12pt]{article}

\usepackage{revy}
\usepackage[utf8]{inputenc}
\usepackage[T1]{fontenc}
\usepackage[danish]{babel}
\pdfoptionpdfminorversion 4

\revyname{Biorevy}
\revyyear{2012}
% HUSK AT OPDATERE VERSIONSNUMMER
% UNDLAD AT SKRIVE I TEMPLATE.TEX - KOPIÉR OG OMDØB I STEDET FOR
\version{2}
\eta{$X$ minutter}
\status{Færdig}

\title{Busch til MUS}
\author{Markus Drag, Biorevy}

\begin{document}
\maketitle

\begin{texxers}
	\texxer{Helene E}[frk.ellyton@hotmail.com]
\end{texxers}

\begin{roles}
	\role{HB}[Markus D] Henrik busch
	\role{Psyk}[Lykke P] Psykolog
\end{roles}


\begin{props}
	\prop{rekvisit}[Hvem skaffer?] 
\end{props}

\begin{sketch}

\scene Henrik Busch kommer ind af døren. Sætter sig på en stol der står midt på scenen. Et skrivebord er sat op i siden, hvor en psykolog sidder bag. HB sidder i en meget udsat position, hvor psykologen sidder og troner bag skrivebordet og tydeligt "har magten".

\says{HB} Hvad drejer dette møde sig om?
\says{Psyk} Jo Henrik, ser du. På Fakultetet afholder vi jo årligt medarbejderudviklingssamtaler. Her tager vi hånd om evt. problemer, spørgsmål og bidrager til medarbejderens personlige udvikling på jobbet.
\says{HB} Javel.
\says{Psyk} Henrik. Lad mig tale lige ud af posen. Jeg er bekymret for dig.
\says{HB}[nedladende] Hvordan?
\says{Psyk} Jeg har undersøgt din baggrund lidt. Lad mig vise dig nogle tegninger som jeg har fået fra din gamle børnehavepædagog.
\says{Psyk} Henrik, fortæl mig hvad der foregår her.
\says{AV}Børnetegning på projektor, hvor en ond tændstiksmand nakker en masse andre tændstiksmænd.
\says{HB} Åh-håhå (små-griner). Han havde glemt forhåndsgodkendelsen.
\act{Børnetegning med tændstiksmand der bliver savet over}
\says{HB} Glemt at vedhæfte dispensation.
\act{Børnetegning med tændstiksmand der bliver brændt på bål}
\says{HB} Glemt studielederens underskrift.
\says{AV}Børnetegning med atombomber
\says{HB} glemt at printe kursusbeskrivelsen ud.
\says{Psyk} Ok ok! Jeg tror jeg forstår. Lad mig prøve en anden kendt associationsøvelse. Jeg viser dig nu et billede, og så fortæller du mig hvad du tror, det er.
\says{AV}Viser et billede af en streg
\says{HB} Hmm. En studerende der ligger fuld af druk uden for Biobar og ikke har bestået en eneste eksamen i fem blokke!
\says{AV}Viser et billede af en kasse
\says{HB} En rustur der er fuldstændig uden faglighed, fyldt med ungdomskriminelle rusvejledere der bare drikker øl og griner af mig!
\says{AV}Viser et billede af en cirkel
\says{HB}[meget sur] En rusvejleder der griner af mig lige i mit åbne fjæs! Jeg skal fandme....!!! \scene HB rejser sig op!
\says{Psyk} Rolig, Henrik rolig!! 
\says{Psyk} Ok, Henrik, jeg tror jeg har fået rigelig bekræftelse i min medarbejdervurdering. Henrik, det er min opfattelse at du desværre ikke er egnet som prodekan! Du hader jo studerende.
\says{HB} Jamen...
\says{Psyk} Jeg kan ikke lade dig fortsætte som prodekan.
\says{HB} Javel.
\says{Psyk} Javel?
\says{HB} Javel.
\says{Psyk} Henrik. Det er okay at reagere følelsesmæssigt. Jeg ved det er et chok.
\says{HB} Ja, det var et chok.
\says{Psyk}[begejstret] Det er godt at høre. Vent, hvad mener du med at "det var"?
\says{HB} Ja, altså. \act{HB rejser sig} Jeg lavede også lidt research. Jeg kunne jo ikke gå til møde uden forberedelse. \act{ondt smil}.
\act{Trækker stolen tættere på skrivebordet}
\says{HB} Jeg søgte lidt i gemmerne på fakultetet og jeg fandt denne her. Ja, det var et chok.
\act{Trækker gammel slidt seddel op}
\says{Psyk} Hmm, hvad er det... En meritoverførsels-blanket...? 
\says{HB} Ja. Det er korrekt. Må jeg bede dig se på navnet.
\says{Psyk} Jamen... Det er jo min blanket! Den må være mindst 20 år gammel!
\says{HB} Ja, det er korrekt. Du tog et OT-kursus på NatFak i 1980.\act{HB sætter sig} Men mere interessant, kan du se nogen underskrift fra studielederen?
\says{Psyk} Næehj... Det har jeg nok glemt. Jeg var jo ung!
\says{HB} Så er den ikke gyldig. Dit kursus er ikke meritoverført, og du har ikke bestået din kandidat.
\says{Psyk} Jamen.... Jamen... Min studieleder er jo død før længst.
\says{HB} Kære psykolog. Det er okay at reagere følelsesmæssigt. Jeg ved det er et chok.
\says{Psyk} Jamen... Jamen..
\says{HB} Men du kan jo prøve at søge dispensation.
\scene HB rækker et A4 ark. Trækker skrivebordet et par meter tilbage, så psykologen sidder helt alene på scenen, ligesom HB sad i starten
Psyk udfylder hurtigt dispensationen, rækker den til HB, stadig meget chokeret
\says{HB} Ahh, hvad har vi her. En dispensationsansøgning? Næh, hvor spændende!
\act{HB læser, meget opstemt}
\says{HB} Jamen... Det var dog forfærdeligt. Nej, nej gud dog, glemt studielederes underskrift... Og nu er studielederen død.. Ikke psykolog ALLIGEVEL. Det var dog forfærdeligt!
\act{HB river papiret i stykker}
\says{HB} Dispensationen er AFVIST!

Alternativ afslutning:
\act{Kalder på samtale-anlægget}
\says{HB} Sekretær, følg venligst denne forvirrede mandsperson ud.
\scene Sekretær følger eks-psykolog ud der tydeligt jamrer og græder




\end{sketch}
\end{document}
