\documentclass[a4paper,12pt]{article}

\usepackage{revy}
\usepackage[utf8]{inputenc}
\usepackage[T1]{fontenc}
\usepackage[danish]{babel}
\pdfoptionpdfminorversion 4

\revyname{Biorevy}
\revyyear{2012}
% HUSK AT OPDATERE VERSIONSNUMMER
% UNDLAD AT SKRIVE I TEMPLATE.TEX - KOPIÉR OG OMDØB I STEDET FOR
\version{2}
\eta{$1.0$ minutter}
\status{Færdig, skal adapteres til scenen}

\title{Dyrere}
\author{Markus Drag, Thomas Bech-Thomassen, Biorevy}

\begin{document}
\maketitle

\begin{texxers}
	\texxer{Niklas S}
\end{texxers}

\begin{roles}
	\role{AU}[Søren HR] Autoritetstro studerende
	\role{UV1}[Niklas S] Uvidende studerende 1
	\role{UV2}[Thomas LH] Uvidende studerende 2
\end{roles}


\begin{props}
	\prop{Kunst}[Hvem skaffer?] Abstrakt kunst
	\prop {Whiteboard/opslagstavle}[Hvem skaffer?] Noget at sætte et stykke papir op på
	\prop{Slide}[Teknik] Slide med teksten "AKB", billede af samme
\end{props}

\begin{sketch}
\says{AV}Slide med teksten "AKB", billede af samme
\scene To piger sidder i en sofa på AKB og terper. UV har tegnet en fin tegning og hun sidder sammen med sin veninde og griner lidt af den. De er enige om at den skal hænges op på væggen, tegningen forestiller deres lærer måske.



\says{UV1}Sådan, der sidder den flot!
\scene AU kommer gående ind over scenen, stopper. Hans ansigt fyldes med indignation og forargelse ligesom på Biocenter. Går hen til de to piger.
\says{AU} Hvad fanden er det i har gang i?!
\says{UV1}Hva'?
\says{AU} Ja, det dér?! Det her er AKB, den her er ikke nær abstrakt nok. Hvor er de psykedeliske farver? Hvad er det egentlig du forestiller dig?
\says{UV1}Jamen, den er da meget fin..?
\says{AU} Ja, men det her er jo ikke din mors køleskab, vel? Det her er Universitetet! Her vil vi KUN have finkultur! Ledelsens motto har jo altid været ?Form over funktion? ?- det skal du ikke komme her og besudle med dine børnetegninger. Føj! 




\end{sketch}
\end{document}
