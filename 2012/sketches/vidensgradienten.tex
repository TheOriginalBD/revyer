\documentclass[a4paper,12pt]{article}

\usepackage{revy}
\usepackage[utf8]{inputenc}
\usepackage[T1]{fontenc}
\usepackage[danish]{babel}
\pdfoptionpdfminorversion 4

\revyname{Biorevy}
\revyyear{2012}
% HUSK AT OPDATERE VERSIONSNUMMER
% UNDLAD AT SKRIVE I TEMPLATE.TEX - KOPIÉR OG OMDØB I STEDET FOR
\version{1}
\eta{$X$ minutter}
\status{Næsten færdig}

\title{Vidensgradienten}
\author{Sarah Brofeldt, Thomas Bech-Thomassen, Biorevy}

\begin{document}
\maketitle

\begin{texxers}
	\texxer{Sarah B}[sbrofeldt@gmail.com]
\end{texxers}

\begin{roles}
	\role{UV}[Julie KS] Underviser
	\role{AV}[AV] A/V
\end{roles}


\begin{props}
	\prop{Slideshow}[Sarah B, Thomas BT] 
\end{props}

\begin{sketch}


\says{UV} KU er som en levende organisme. Vi har skabt et miljø, hvor viden og andre positive elementer findes i høj koncentration

\says{AV} Slide med studerende som celle med viden$^+$ og studieaktivitet$^+$ udenfor celle, med transportere i membran som aktivt transporterer disse ind.

\says{UV} Desværre må vi erkende, at mange studerende ankommer i en tilstand, som ikke er favorabel for denne overførsel, og i nogen tilfælde er de nødvendige transportere og poriner ganske enkelt ikke til stede. I mange af de undersøgte russer, så situationen faktisk således ud

\says{AV} Slide med ingen transportere (evt. overstregne) i membranen, og en høj intracellulær koncentration af livsglæde$^+$.

\says{UV} Ved at påvirke den studerende på den rette måde, er det imidlertid muligt at aktivere en antiporter, som ved brug af energi pumper livsglæde$^+$ ud af den studerende til fordel for studieaktivitet$^+$.

\says{AV} Slide af antiporteren

\says{UV} Når den intracellulære koncentration af studieaktivitet$^+$ er tilstrækkeligt høj, opreguleres dannelsen af studiemiljøkomplekser bestående af studierum, studiebar og læsepladssubunits i membranen.

\says{AV} Slide med høj koncentration af studieaktivitet$^+$ i celle og studiemiljøkompleks i membranen.

\says{UV} Disse komplekser fungerer som aktive transportere, således at viden$^+$ kan føres mod gradienten og ind i den studerende. 

\says{UV} Denne kaskade er utroligt effektiv, så længe det extracellulære miljø er favorabelt. Mange studerende forsøger imidlertid at optimere processen ved at udnytte øl$^-$'s tendens til at binde sig til frie, positive elementer, og dernæst diffundere over membranen.

\says{AV} Slide (det tager vi et foto af, det skal være nogle øl på en bardisk med en custom-etiket med ``vidste du...'' og nogle forskellige biologiske fakta. Udkast til etiketter påskønnes.)

\says{UV} De glemmer desværre, at udover øl-viden-komplekset, dannes der andre neutrale ølkomplekser som også diffunderer frit over membranen \ldots

\says{AV} Slide: Mange neutrale ølkomplekser udenfor cellen, f.eks. "`øl-meme"', "`øl-overspring"', "`øl-spil"' etc. - alle med pile ind mod cellen. Mange pile, gerne rodet! I cellen er der kun "`øl-viden"' med pile ud af cellen. 

\says{UV} \ldots hvorfor nettoeffekten blot er, at den studerende tilegner sig tilfældig information på bekostning af tilfældig \emph{studierelevant} viden.

\says{AV} Slide: Vidste du at hunde ikke kan se opad?

\says{UV} Konsekvensen af dette er en signifikant forlængelse af studietiden.

\says{AV} Slide: Graf med studietid hhv. og uden øl (to grafer i et plot. Laaang tid med øl.)

\says{UV} Der er mange muligheder for at nedregulere den uønskede ølkompleksdannelse, men den simpleste og sikreste er at hæmme tilførslen af extracellulært øl$^-$, f.eks. ved brug af studieaktivitetskrav.

\says{UV} Hvis ikke øltilførslen hæmmes, risikerer man ultimativt at øl$^-$ bindes til alle essentielle proteiner hvorefter checkpoints ikke kan passeres, og cellen går i G$_0$.

\says{UV} Normalt vil cellen dog være i stand til at gennemføre sin livscyklus og gradvist modnes til en kandidatcelle. Kandidatceller er pluripotente, og kan bl.a. specialisere sig til Ph.D, forsker, eller konsulentceller. Det er dog langt mere sandsynligt, at cellen ender i en terminal jobcenterfase.

\says{UV} Hermed er der redegjort for en studiecelles overordnede livscyklus. Der har i tidens løb været meget kritik af universitetssystemet, men det må være åbenlyst, at det er en naturlig konsekvens af disse pathways, at livsglædesniveauet må falde omvendt proportionelt med den endelige koncentration af viden.

\end{sketch}
\end{document}
