\documentclass[a4paper,12pt]{article}

\usepackage{revy}
\usepackage[utf8]{inputenc}
\usepackage[T1]{fontenc}
\usepackage[danish]{babel}
\pdfoptionpdfminorversion 4

\revyname{Biorevy}
\revyyear{2012}
% HUSK AT OPDATERE VERSIONSNUMMER
% UNDLAD AT SKRIVE I TEMPLATE.TEX - KOPIÉR OG OMDØB I STEDET FOR
\version{3}
\eta{$4.00$ minutter}
\status{Færdig}

\title{Eksamen i DK Fauna}
\author{Markus Drag, Biorevy}

\begin{document}
\maketitle

\begin{texxers}
	\texxer{Thomas BT}[thomas.bech.thomassen@gmail.com]
\end{texxers}

\begin{roles}
	\role{CE}[Søren PH] Censor
	\role{Stud}[Thomas LH] Studerende
	\role{AV}[AV]
\end{roles}


\begin{props}
	\prop{Slideshow}[Markus D] 
	\prop{Lineal}[Thomas LH]
	\prop{Skilt}[Thomas LH] Skilt: 02
\end{props}

\begin{sketch}

\textbf{Rækkefølge af bogstaver: M S R H A E T U K H Å}
\\

\says{AV}Tekst på projektor: Eksamen i Danmarks Fauna - SLIDE 1

\scene Stud kommer ind.

Bandet spiller "Led zeppelin - Moby Dick"

\scene Studerende er glad, oppe og køre og selvsikker. 

\scene På eksamensbordet står det dyr, han skal nøgle. Dette er vist som en plakat af dyret sat på en opslagstavle. Det er dækket af et sort klæde.


\scene Den studerende går nogle omgange omkring eksamensbordet, indtil musikken stopper. Derefter går en timer-agtig musik i gang (tænker noget alla den lyd fra hvem vil være millionær), men bare med mulighed for at gå hurtigere senere.

 Han fjerner så det sorte klæde, der hænger for det han skal nøgle: kradser sig så i håret og ser spørgende ud (han ved ikke hvad det er): Der skal derfor nøgles.
 
\says{AV}SLIDE 2 + 3.

\scene Han slår op i bogen: På storskærmen kører der så nøglen imens, så publikum kan følge med (tænkte at billedet af dyret måske kunne være sort hvidt, for at gøre det endnu dummere). Han peger op mod stormskærmen for at angive sine valg. 

\says{AV}NØGLEPUNKT 1: (Den studerende er selvsikker) - SLIDE 4

\scene Den studerende vælger Pattedyr. Dette giver ord M.

\says{AV}NØGLEPUNKT 2: (Den studerende er fortsat selvsikker) - SLIDE 5

\scene Den studerende vælger at den har ører. Dette giver ord S.

\says{AV}NØGLEPUNKT 3: (Den studerende ved at alle pattedyr er supercute!) - SLIDE 6

\scene Valgmuligheder er om den ser sød ud. Dette giver ord R.

\says{AV}NØGLEPUNKT 4: (Den studerende ser nu forvirret ud) - SLIDE 7

\scene Længde er relativt lang. Dette giver ord H.

\says{AV}NØGLEPUNKT 5: (Vender sig om mod publikum og ser spørgende ud) - SLIDE 8

Farve er lorte brun. Dette giver ord A.

\says{AV}NØGLEPUNKT 6: 
\scene Viser sin fortvivlelse åbenlyst, holder sig for øjnene og gætter så - \says{AV}SLIDE 9

\scene Lugt er genkendelig. Dette giver ord E.

\says{AV}NØGLEPUNKT 7: (Ser meget opgivende ud og kaster sine hænder mod jorden) - SLIDE 10

\scene Kranie: vælger ganebenets bagkant jævn afrundet. Dette giver ord T.


\says{AV}Opsummering - SLIDE 11
\scene Den studerende fik bogstaverne M, S, R, H, A, E og T. I slideshowet arrangeres bogstaverne ved animation. Bum-bum-bum hvad giver det... 

\says{AV}H A M S T E R!
FORKERT - Neeeej, det passer ikke! Om igen.

\says{AV}Afspil denne forkert-lyd: http://www.youtube.com/watch?v=fNhh0IjcroA

START FORFRA

 Baggrundslyden begynder at tælle hurtigere
\scene Den studerende må nu skynde sig, og viser det i sit kropsprog. Han nøgler igen 

\says{AV}NØGLEPUNKT 1: \scene (Den her er han sikker på. Ser utålmodig ud og skynder sig videre)

\scene Dette giver ord M.

\says{AV}NØGLEPUNKT 2: \scene (Endnu engang sikker, og skynder sig videre)

Dette giver ord S.

\says{AV}NØGLEPUNKT 3: \scene (Siger ?PFF? og ser ud som om det er totalt åbenlyst)

Dette giver ord R.

\says{AV}NØGLEPUNKT 4: \scene (Trækker på skuldrene og gætter på noget andet)

Dette giver ord U.

\says{AV}NØGLEPUNKT 5: \scene (Tiden tikker nu hurtigere, og den studerende går fra frustration til panik)

Dette giver ord K.

\says{AV}NØGLEPUNKT 6: \scene (Forsøger sig med noget andet)

Dette giver ord H.

\says{AV}NØGLEPUNKT 7: \scene (Skynder sig også bare at vælge noget)

Dette giver ord Å.

\says{AV}OPSUMMERING: - SLIDE 19

M, S, R, U, K, H, Å.

\says{AV}Animeret overgang:

K U S H Å R     + M

\scene Den studerende bliver en anelse befippet, men prøver at ændre det.

\says{AV}Animeret overgang:

H U S M Å R     + K

 TIDEN ER LØBET UD!

\scene Censor kommer ind. 

\scene  Den studerende er panisk, da han stadig har et K stående udenfor. Han ses derfor meget befippet med nøglen. Han forsøger at fjerne K'et ved at "smide det ud".

\says{AV}Animeret overgang:

	H U S M Å R  og K'et flyver ud til siden i PowerPoint slidet




\end{sketch}
\end{document}
