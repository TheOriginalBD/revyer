\documentclass[a4paper,12pt]{article}

\usepackage{revy}
\usepackage[utf8]{inputenc}
\usepackage[T1]{fontenc}
\usepackage[danish]{babel}
\pdfoptionpdfminorversion 4

\revyname{Biorevy}
\revyyear{2012}
% HUSK AT OPDATERE VERSIONSNUMMER
% UNDLAD AT SKRIVE I TEMPLATE.TEX - KOPIÉR OG OMDØB I STEDET FOR
\version{4}
\eta{$X$ minutter}
\status{Færdig}

\title{En Barjoke(Antistofsketchen)}
\author{Markus Drag, Biorevy}

\begin{document}
\maketitle

\begin{instructors}
  \instructor[Julie KS]
\end{instructors}

\begin{texxers}
	\texxer{Thomas BT}[thomas.bech.thomassen@gmail.com]
\end{texxers}

\begin{roles}
	\role{FY}[Thomas BT] Fysiker
	\role{BI}[Isabel HR] Biolog
	\role{VO}[Niklas S] Voice over
	\role{AV}[AV]
\end{roles}


\begin{props}
	\prop{Fysikerkittel}[Hvem skaffer?] Evt. med strålingstegn?
	\prop{Biologkittel}[Hvem skaffer?] Evt. med biolog-niceness?
	\prop{Bar}[Rekvisitrum] Caféen?
	\prop{FeZ}[Rekvisitrum] FysikerFeZ
	\prop{Køletaske}[Julie KS] Køletaske
	\prop{Tøris}[Hvem skaffer?]
	\prop{Ringetone}[A/V] Ringetone til mobiltelefon
\end{props}



\begin{sketch}


\scene Biolog og fysiker mødes på scenen, stiller sig ved baren.
\says{VO} En fysiker og en biolog kommer ind på en bar.
\says{BI} Puha, hold kæft en hård dag i lab. 
\says{FY} Nå, hvad har du lavet? 
\says{BI} Jeg har brugt hele dagen på antistoffer. Det var ikke min bedste dag i dag, men jeg fik da lavet lidt.
\says{FY} Antistof? Det lyder utroligt interessant? 
\says{BI} Ja, det er sgu meget cool. Anvendelsesmulighederne er astronomiske.
\says{FY} Ja, det er helt utroligt. En opskalering af produktionen ville kunne revolutionere Verdens energiproduktion! 
\says{BI} Jaeh. Det er jo egentlig ikke sååå svært at frembringe. Jeg har haft lidt mandagssyge i dag, og jeg endte med 10 microgram eller sådan noget. 
\says{FY} 10 microgram!? Det er jo fantastisk! Hidtil har vi fysikere kun kunne fremstille 11 nanogram - og det har højst holdt i tusind sekunder!
\says{BI}[undrende] Putter I dem ikke på frys? Det skal man altid huske! Som jeg plejer at sige; på frys med det samme og sølvpapir om.
\says{FY} På frys?! Det skal da varmes op til flere tusind kelvin!
\says{BI}[Lidt arrogant] Hør nu her, jeg ved ikke hvilken laborant I har ansat -- men uden kulde degenerer jeres antistoffer hurtigt.
\says{FY} Så det errr\ldots ligesom i kold fusion? Virker det virkelig? Det har vi aldrig tænkt på. Hvad med jeres acceleratorer? Vi kører på ca. lysets hastighed.
\says{BI} Vi kører ved 1000 G. i 20 min -- det er lige tid til en kop kaffe.
\says{FY} Det er jo utroligt. Og kun i 20 min? Vores sidste beregninger har vist at det vil ta' os 10.000 år at lave 10 microgram!
\says{BI}[Meget forundret] 10.000 år? Så længe har jeg da ikke tid til at vente! Mine celler skal lyse! Men hør nu her; Hvis du er så interesseret i antistoffer, så har jeg da masser på lager. Faktisk har jeg da min seneste ekstraktion med.\act{Finder PCR tube op med antistof fra sin køletaske med tøris i}
\says{BI} Du kan da få en prøve, hvis du er så tændt på det\ldots \act{for sjov} du kan jo udgive en artikel om det\ldots
\says{FY} Jaeh. Altsåeh\ldots jeg har jo ikke lavet det selv, meeeenæeh, antistof alligevel. Jo! Jeg gør det! Giv mig noget antistof!
\says{BI} Okay, fedt mand. Her er en prøve -- og husk: første gang er gratis.
\says{FY} Åhr, hvad! Bare vent til hovederne ovre på NBI ser det her -- de bliver ellevilde!\act{Fysikeren går ud}
\says{A/V} Biologens mobiltelefon ringer
\says{BI} Ja, hej, det er Ole M, hvad kan jeg gøre for dig? \ldots Kernefusion, siger du? Jaja -- det kan vi klare ved 37 grader\ldots

\scene Lys ned


\end{sketch}
\end{document}
