\documentclass[a4paper,11pt]{article}

\usepackage{revy}
\usepackage[utf8]{inputenc}
\usepackage[T1]{fontenc}
\usepackage[danish]{babel}


\revyname{Biorevy}
\revyyear{2013}
\version{0}
\eta{$n$ minutter}
\status{Ikke færdig}

\title{Nøglevisen}
\author{Markus B}
\melody{T. Egner, H. Rasmussen: ``Vi lister os afsted på tå''}

\begin{document}
\maketitle

\begin{roles}
\role{B1}[Frederikke L] Biolog 1
\role{B2}[Karla MJ] Biolog 2
\role{BV}[Markus B] Biolog Væmmelig
\role{INS}[Marcus G] Instruktør
\end{roles}

\begin{song}
%  \sings{S} Denne sang rimer knapt

%  \sings{S}[omkv] Ringe agt

S1: Vi lister os afsted på tå 
S1: når vi skal ud og nøgle.
S2: Vi nøgler alt på må og få
S2: en fugl, en fisk, en øgle.
S3: Jeg nøgler som de bedste kan
S3: Ved brug af blot min tissemand 
S1: Med læder og poter og pels og med tand
S1: Frederikke S2: Og Karla S3: Og Sodoman 

\act {B1 \& B2} Stiller spørgsmål ved "tissemand" linjen. Skulle have været "net og spand" 

S1: I jagt og sport dér går vi ind 
S1: for vi skal fange ugler.
S2: Forretningen for os med hang 
S2: til fjer og 
S3: -'Fugle-Kugler'
S3: Ja, uglens hals er smidig nok
S3: Så gi du den et ordentlig gok
S3: Og stir den så ned og ja gi' den så nok
S3: Så den vågner med et akut penis-chock

\act {B1 \& B2} udvandrer i protest

\sings{BV}
Hvepse de er ækle dyr 
men sikken dog en talje
Brod og gift og smerter ja, 
er nu blot en detalje
Mit lem det har jeg pakket ind
Stål-uld og hygiejnebind.
Så nu kan jeg frit gå lidt over gevind
Ja fat hvepsebo, penis, og tryk den ind!

Hesten den er stor og stærk, 
og findes hér i mængde
men heste-verdenen er træls, 
og handler kun om længde
Ja hingstene de blærer sig, 
jeg føler at de ler af mig.
Og lidt jalousi har det vækket i mig,
Så mon ikke lasagnen bli'r ekstra sej

Jeg kasted' mig på dyrene 
For jeg fik ingen kvinder
Med kradsetræ og foderbræt 
Var alle mand en vinder
Men et problem skal fixes her
Hos dyr'ne du'r jeg ikke mer'
Jeg sidder, jeg kigger, jeg lyster og ler
Mens jeg venter på bør'ne får frikvarter


\end{song}

\end{document}

