\documentclass[a4paper,11pt]{article}

\usepackage{revy}
\usepackage[utf8]{inputenc}
\usepackage[T1]{fontenc}
\usepackage[danish]{babel}


\revyname{Biorevy}
\revyyear{2013}
\version{2}
\eta{$5$ minutter}
\status{Næsten færdig - versefødder rettes}

\title{Yndlingsfeltstation}
\author{Markus D}
\melody{Klumben feat. Raske penge: ``Faxe kondi''}

\begin{document}
\maketitle

\begin{roles}
\role{S1}[Markus D] Sanger 1
\role{S2}[Andreas N] Sanger 2
\role{D1}[Helene E] Danser 1
\role{D2}[Isabel HR] Danser 2
\role{INS}[Marcus G] Instruktør
\end{roles}

\begin{song}


%  \sings{S}[omkv] Ringe agt
  \sings{S}2 x Omkvæd:
For det min yndlings feltstation
På togbilletten har jeg ikke spildt en krone
Jeg nyder hver en dag
Og jeg vil ikke hjem
Men efter sommer
starter blokken jo igen

Vers 1:
Det revysten her igen
jeg savner fandme Kristiansminde
Jeg bli'r tiltrukket af den bolig
som var det en fræk kvinde
Og det er ik’ for sjov, når jeg siger-
jeg har brugt- ordentlig mange penge
På at komme til det sted, hvor jeg føler mig hjemme
Det siger (øllyd)) for det er jo ren feltbiologi
Min Kristiansminde tur nydes med masser drikkeri
Min vejleder, han siger jeg skal stop'
Men Kristiansminde points, de slipper aldrig op
Og nu er tilmelding åben, så jeg gør hvad der passer mig
Et Kristiansminde kursus jo tak det er den rigtig' vej
For skal jeg endelig knokle, er der kun en ting der dur
Kristiansminde gi'r mig lyst - til at nøgle mor natur
	
Vers 2:
Det ligger ude i en skov, det er Kristiansminde
Du ska' spørge Ulrik om lov, det er Kristiansminde
Du ska' lave en svær opgav', det er Kristiansminde
Male et værelse eller hugge no'ed brænde
Jeg kan godt li' Salten Skov og stationen i Hillerød
Men der kun et sted hvor jeg vil begraves når jeg er død
Kald det slidt hus, kald det for-forældet lort
Kristiansminde er mit eget Himmeriges port
Den har mug, skimmelsvamp og sjove ting i min mad
Det er et vidunder  lige meget hvad
Kristiansminde har alt en biolog skal bruge
Laboratorium, nøglebøger, kraftig mikroskop
løber du tør for øl, så gå i kælder'n knægt
Vi mangler tænkevand til at nøgl' det insekt
Vor organisme viden uddeler smæk
Ta' til Sorø, det er altid korrekt

2 x omkvæd

Vers 3:
Jeg drømmer om Kristiansminde hver en morgen i Lundbeck
Drømmer om den dag hvor jeg bar' li kan tage toget væk
Jeg ta'r på Kristiansminde selv når jeg har en date med en pig'
For kærlighed slås af ægte felt-bio-logi
Biologen han ka' leve på feltkursus året rundt
Selv min læge si'r stop! det ikke er særlig sundt
Men på felttur ska jeg ikke bruge medicin
Alle felt-skader burde kunne klares med kirsevin
Spørg: Kristiansminde er du sikker du vil det?
Så siger jeg JA JA ligesom Anders Priemé
Og når jeg skal hente SU så får jeg ingen peng'
Bruger dem på togbilletter så jeg ka' komme hjem!

Vers 4:
Hey yo, jeg håber du forstår hvad vi si'r med denne sang
Originale Felttur-fans siger vinteren bli'r lang
Men biologen her er smart, for han har noget i gærde
Når savnet bliver for stort, her er noget at lære
Find nøglebogen frem, og dine tørret græshopper
Jeg sværger til biologi li' til mit hjerte stopper
Selv før jeg var rusling, kiggede jeg under de brosten
Hvis der er en løbebille, håber jeg det er en god en
For når studiet det er hårdt
Jeg brug'r mit rejsekort
ta'r til Kristiansminde og stanger snapsen som en hjort
hvis dagen er lidt trist
Jens Høegh gi'r dig fist
så find din indre styrke som en Kristiansminde-ist
Hvis du kommer til min by
Skal du se Biobaren og den Biorevy
Og når du er der, og chiller med begge ting
Så ved du Sorø venter inden alt for læng'

2 x omkvæd
\end{song}

\end{document}

