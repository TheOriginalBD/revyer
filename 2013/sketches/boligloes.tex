\documentclass[a4paper,12pt]{article}

\usepackage{revy}
\usepackage[utf8]{inputenc}
\usepackage[T1]{fontenc}
\usepackage[danish]{babel}
\pdfoptionpdfminorversion 4

\revyname{Biorevy}
\revyyear{2013}
% HUSK AT OPDATERE VERSIONSNUMMER
% UNDLAD AT SKRIVE I TEMPLATE.TEX - KOPIÉR OG OMDØB I STEDET FOR
\version{1}
\eta{$5$ minutter}
\status{Færdig -- skal lige tilpasses lidt, noget af sketchen hænger ikke sammen}

\title{Boligløs-lejr}
\author{Jennie, Helene, Valdbjørn, Drag, Stine, Nathalia, Mia}

\begin{document}
\maketitle


%\begin{texxers}
%	\texxer{Navn}[email@exmpl.com]
%\end{texxers}


\begin{roles}
	\role{VO}[Thomas BT]Voice over
	\role{BL1}[Jean ES]Boligløs 1
	\role{BL2}[Mia H]Boligløs 2
	\role{BL3}[Regitze P]Boligløs 3
	\role{BL4}[Søren PH]Boligløs 4 -- Kommer løbende ind på scenen
	\role{R}[Jennie B]Ulla Terkelsen?
	\role{ES1}[Anna S]Ekstra studerende der kommer op igennem et telt
	\role{ES2}[Magnus HA]Ekstra studerende der kommer op igennem et telt
	\role{ES3}[Marcus G]Ekstra studerende der kommer op igennem et telt
	\role{ES4}[Søren HR]Ekstra studerende der kommer op igennem et telt
	\role{ES5}[Thomas BT]Ekstra studerende der kommer op igennem et telt
	\role{ES6}[Alina J]Ekstra studerende der kommer op igennem et telt
	\role{INS}[Stine T] Instruktør
	\role{INS}[Thomas BT] Instruktør
\role{Speak}[Thomas BT] Voice over
\end{roles}


\begin{props}
	\prop{Telt}[Hvem skaffer?] Et lille telt der står over lemmen.
\end{props}

\begin{sketch}


\scene Spot på reporter på scenen. Tre boligløse står forhutlet ved et mikrotelt i hjørnet.

\says{VO} Igen i år er der kommet en uventet bølge af boligløse studerende efter studiestart i flere større byer. Ved Universitetsparken på Nørre Campus har Sofaformidlingen valgt at oprette en boligløslejr for at imødegå dette problem. Vi stiller nu om til vores rapporter med Ulla Terkelsen.

\says{R} Er jeg på?...  

\says{VO} Ja, Ulla.

\says{R} Går jeg klart igennem?

\says{VO} Ja, Ulla.

\says{R} ...Ja?... Godaften, Thomas. Jeg står nu i boliglejren på Nørre Campus, hvor de hjemløse studeremnde har mulighed for at slå deres telte op.

\scene Lys på resten af scenen og de boligløse.

\says{R} Hvorfor bor I her?

\says{BL1}Vi har simpelten ikke kunnet finde noget. Der er alt for hård selektion på boligmarkedet til at vi kunne klare os! 

\says{BL2}Tja, og så er det forhåbentligt bare midlertidigt.

\says{R} Men prøver I ikke engang at finde noget andet?

\says{BL1}JO! Jeg var ude at se på den fedeste lejlighed. Der var kun skimmelsvamp i 2(!) vægge, og flere af vinduerne havde stadig glas!

\says{BL2} Jeg var også ude at se på en! Det var den PERFEKTE lejlighed for en akvatisk økolog som mig. Der var glas i alle vinduerne, der var et komfur, og der var endda altid vand på gulvet så mit bachelorprojekt ville blive holdt i naturlige omgivelser i hjemmet!

\says{R} Hvorfor tog du den så ikke?

\says{BL2}Altså, den gik til en af de indfødte. De har ligesom fortrinsret.

\says{R} De indfødte?

\says{BL2}Ja, altså -- dem, der er født ind på ventelisterne!

\says{BL3}Jeg har også set på et kollegieværelse, der kun skulle deles med ti andre - det kostede kun 5000 kroner inkl. kropsvarme!

\says{R} Nådada\ldots
\says{R} Hvordan er livet så her i lejren?

\says{BL1}Tja, varmeristen var taget inden jeg ankom, så det er lidt småkoldt om natten. Det er de kandidatstuderende der har taget de bedste pladser - de har efterhånden været her så længe at de fortjener det.

\says{BL1}Men altså, det er ikke sååå slemt. Man kan endda bade inde på biocenter, hvis man kan komme ind - jeg er dog kun andetårsstuderende så jeg har ikke fået mit studiekort endnu.

\says{R} Er der slet ikke noget godt at sige om at bo her?

\says{BL1}Tjoooh, det er ret tæt på Caféen?\ldots

\scene Boligløs studerende no 4 kommer ind på scenen!

\says{BL4} Har I hørt det?! Der er et ledigt værelse på Grønjordskollegiet!

\says{BL2} Er det selvmordskollegiet?

\says{BL4} Ja!

\says{BL2}[entusiastisk]Der vil jeg herre gerne bo!!! 

\says{BL4}[hektisk] Der er ansøgningsfrist i aften, og det er først til mølle!

\scene Studerende kigger på hinanden og begynder at strømme ud af teltet\\
(op gennem lemmen) og løber ud. De studerende kæmper sig ud af scenen for at komme først til kollegiet.\\

\scene Den sidste studerende der kommer ud, tager teltet og løber med det.




\end{sketch}
\end{document}