\documentclass[a4paper,12pt]{article}

\usepackage{revy}
\usepackage[utf8]{inputenc}
\usepackage[T1]{fontenc}
\usepackage[danish]{babel}
\pdfoptionpdfminorversion 4

\revyname{Biorevy}
\revyyear{2013}
% HUSK AT OPDATERE VERSIONSNUMMER
% UNDLAD AT SKRIVE I TEMPLATE.TEX - KOPIÉR OG OMDØB I STEDET FOR
\version{0}
\eta{$X$ minutter}
\status{Færdig}

\title{Wilhjelm-konferencen}
\author{Markus D, Marcus G, Markus B, Jakob Valdbjørn}

\begin{document}
\maketitle


%\begin{texxers}
%	\texxer{Navn}[email@exmpl.com]
%\end{texxers}


\begin{roles}
	\role{Konf}[Mia H]Konferencier
	\role{F1}[Anna S]Forsker 1
	\role{F2}[Signe S]Forsker 2
	\role{F3}[Søren PH]Forsker 3
	\role{JCS}[Jakob VO]Jens-Christian Svenning
	\role{INS}[Andreas N] Instruktør
\end{roles}


\begin{props}
	\prop{Pult}[Hvem skaffer?]Konference-pult, m påskrift 'Wihjelm+12'
	\prop{Stole}[Hvem skaffer?]4 stole
	\prop{Slideshow}[Hvem skaffer?]Slideshow med billeder af biotoper.
\end{props}

\begin{sketch}



 Wilhjelm konferencen -- seriøst og biologisk.
\\
\textbf{Bemærkninger til Teknikken:}
A/V slideshow! Vi laver selv!
\\


Hvem er Jens-Christian Svenning? LÆS: http://www.dr.dk/Nyheder/Indland/2013/03/05/082134.htm


"  Med en kreativ forvaltning i nogle store naturområder, som for eksempel nationalparker, kunne elefanter godt være en mulighed. Jeg synes, at der er gode muligheder for at få en mere righoldig stordyrsfauna igen, siger han til metroXpress." -- Jens-Christian Svenning.

Wilhjelm-konferencen 2013 byder på årets store emne -- nemlig rewilding. Genudsætning af store pattedyr, som oftest planteædere eller top-prædatorer med det formål at skabe mere naturlige og oprindelige biotoper med dertil økologisk regulering.
Dog har forskeren fra AAU, Jens-Christian Svenning, meget større planer for DKs natur, han er nemlig VILD med elefanter!

A/V slideshow: Velkommen til Wilhjelm+12!

\says{Konf} Velkommen til dette års Wihjelm-konference! Årets tema er som bekendt Rewilding -- genudsætning af store pattedyr som en del af bestræbelserne på at genskabe de oprindelige økologiske dynamikker i vort menneske-prægede økosystem. Vi skal snakke vildsvin, bisonokser, bævere og ikke mindst -- de fantastiske ulve!

\act{Forskere klapper}

\says{Konf} På Klosterheden i Jylland har vi haft stor succes med udsætning af tyske bævere, der har skabt store fluktuationer i fødetilgængeligheden.

\says{F1} Spændende!

\says{F2} Fantastisk dynamik.

\says{F3} Utroligt samspil!

\says{Jens-Chr} Mjaa.... Næææh... Det er godt nok lidt småt, sådan bævere.

\says{Konf} Men indflydelsen er enorm! Bare en enkelt familie af bævere påvirker naturen lige så meget som en elefant!

\says{Jens-Chr} Stort, påvirkning -- elefanter....

\act{Jens-Chr sidder og tænker grundigt over elefanter, er meget tankefuld.}

\says{A/V slide} Billede af Thy klitplantage.

\says{Konf} Vi skal også snakke om årets 'hotteste' biotop, nemlig Thy Klitplantage. De fleste har hørt om apex-prædatoren Canis lupus der ved egen indvandring er en interessant tilføjelse...

Jens-Christian har hånden oppe -- ivrig!

\says{Konf} Øh ja, professor Svenning?

\says{Jens-Chr} Øh jo, altså [nørdet stemme], nu ser jeg på på på billedet her, og så tænker jeg -- der er rigtigt meget plantemateriale. Og og og, er der et dyr, der elsker planter, bark og blade, jamen så vandrer mine tanker pludselig hen på - elefanter.

\says{Konf} Elefanter?

\says{Jens-Chr} JA! Elefanter! 

\says{Konf} Du vil have elefanter?

\says{Jens-Chr} Ja!

\says{Konf} I Thy?

\says{Jens-Chr} JA! Perfekte biologiske omsætnings-maskiner til stærk forøgelse af den biologiske dynamik! Og tænk sig engang [bliver mere omsorgsfuld], de dejlige snabler overalt! Og Jylland -- det er stort! Et stort land kræver store dyr! Elefanter!

\says{Konf} Okay, jamen den kan vi lige notere os.... 

Der hviskes i krogene.

\says{Konf} Næste biotop er dette kombineret kultur- og naturområde, nemlig Amager Fælled. Præget af pionerarter og menneskelig kontakt kan dette være optimal habitat for udsætning af en række fuglearter og --

\act{Jens-Chr vifter vildt med hånden.}

\says{Konf} Prof. Svenning?

\says{Jens-Chr} Øh ja ja altså -- nu tænker jeg bare højt. Haha ja, jeg får nemlig en helt vild idé. Øh ja, altså -- er der et dyr der rigtigt godt kan li' menneskelig kontakt, ja netop elsker det, og slet ikke har noget imod fugle, jamen øh -- så tror jeg at vi skal over i noget rigtigt stort igen!

\says{Konf} Hvilket stort dyr tænker du på? Vi ved at landskabet har været præget af hjorte tidligere, er det dem du tænker på?

\says{Jens-Chr} Nejnejnejnejnej! Endnu større! Ja, et dyr der elsker mennesker, så der vil jeg foreslå -- elefanter! 

\says{Konf} Du vil ha' elefanter?

\says{Jens-Chr} Ja!

\says{Konf} På Amager fælled?

\says{Jens-Chr} Ja, dejlige elefanter, over hele Amager Fæll- NEJ! [kunstpause] -- Over HELE Amager -- ja ja ja! 

\says{Konf} På Christiania?

\says{Jens-Chr} Jaa!

\says{Konf} I lufthavnen?

\says{Jens-Chr} Ja, der har de jo masser af plads!

\says{Konf} Hvad med KUA?

\says{Jens-Chr} Ja, der er jo plads til et kæmpe reservat!

\says{F1} Det er uacceptabelt, det vil ganske enkelt skabe konflikter!

\says{F2} Ja -- og elefanter er tilpasset et ækvatorielt tropisk klima! Amager er KOLDT!

\says{F3} Og hvad med alle idéhistorikerne på KUA?

\says{Jens-Chr} Neeeeej, vrrøvl, folk fra Amager elsker store dyr. Tænk på alle kamphundene! Og vi kan da altid bare varme amager op!

\says{Konf} Mine herrer! Må jeg bede om ordet, tak. Den sidste biotop er denne marine kyststrækning hvor marsvin og sæler boltrer sig. Vi vil diskutere udsætningen af en række marine oprindelige arter som for eksempel

\act{Jens-Chr vifter vildt med hånden.}

\says{Konf} (dybt suk) Svenning...

\says{Jens-Chr} Ja altså, det er jo et smukt stykke vand her, og stort! Jeg tænker -- jeg tænker..... Krebs og muslinger. Sandstrand. Badende børn, glade familier der nyder den danske sø. Sø. Søpindsvin? Søheste? Søløver. Løver? ELEFANTER!



\end{sketch}
\end{document}