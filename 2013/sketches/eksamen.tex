\documentclass[a4paper,12pt]{article}

\usepackage{revy}
\usepackage[utf8]{inputenc}
\usepackage[T1]{fontenc}
\usepackage[danish]{babel}
\pdfoptionpdfminorversion 4

\revyname{Biorevy}
\revyyear{2013}
% HUSK AT OPDATERE VERSIONSNUMMER
% UNDLAD AT SKRIVE I TEMPLATE.TEX - KOPIÉR OG OMDØB I STEDET FOR
\version{0}
\eta{$X$ minutter}
\status{Skal skrives sammen med DNA-sangen}

\title{ Skriftligt eksamens-stumfilm}
\author{Biorevy}

\begin{document}
\maketitle


%\begin{texxers}
%	\texxer{Navn}[email@exmpl.com]
%\end{texxers}


\begin{roles}
	\role{Eks}[Anna S]Eksaminant
	\role{Vagt}[Thomas H]Gammel eksamensvagt
	\role{DNAfe}[Signe S]DNA-fe
	\role{INS}[Jennie B] Instruktør
\role{AV}[AV] A/V	
\end{roles}


\begin{props}
	\prop{Fe-kostume}[Hvem skaffer?] Noget med DNA?
\end{props}

\begin{sketch}

\emph{\textbf{OBS: skal skrives sammen med DNA-sangen}}\\


\textbf{Bemærkninger til Teknikken:}
A/V musik af forskellige kompositioner. Vi laver selv slideshow med lyd.



En studerende skal til skriftlig eksamen der dog afspilles som en stum-film med tilhørende musik-
lydspor med klassisk musik. Der skal være meget gestikuleren og krops-mimik (det er det sjove i denne sketch).

På A/V kører tekst som på en klassisk stum-film med fine indramninger der beskriver handling og følelser!

Lys op.

\says{AV}Det bli'r en spændende eksamen i molekylær biologi idag. Jeg er glad og forventningsfuld!
Musik: Glad klassisk komposition.

Eksaminant kommer ind og er meget glad. Sætter sig ved bordet, strækker hænderne, knækker fingre. Er klar!

Eksamensvagt giver eksaminant opgaven.

\says{AV}Uha, nu bliver det spændende.
Musik: Forventningsfuld, trommehvirvel.

Eksaminant åbner opgave-arket, men finder ud af at spørgsmålene er meget svære.

\says{AV}Jamen... jamen...
Musik: Spændingsopbyggende.

Eksaminant kigger op.

\says{AV}Jamen, det er jo slet ikke hvad jeg troede det var!
Musik: Udløsende, begynder at blive sørgelig.

Eksaminant tydeligt skuffet. Opgivende.

\says{AV}Jeg troede da ikke at disse spørgsmål kom med! Det var da ikke pensum!.
Musik: Benægtende, lidt vredt.


Eksaminant benægter at spørgsmålene er en del af pensum. Rækker hånden op for at tilkalde eksamensvagt.

Eksaminant spørger nu eksamensvagt ved gestikuleren.

\says{AV}”Der må være sket en fejl. Lac-operonen er så sandelig ikke en del af pensum i Almen Molekylærbiologi”.
Musik: Samme.

Eksamensvagt nikker, kigger, nikker igen forstående.

\says{AV}Jeg er jo bare eksamensvagt, det ved jeg ikke noget om. Men hvis du er heldig kan det jo være at DNA-feen kommer på besøg...”. \act{Eksamensvagetn griner ondt/sarkastisk/arrogant (mimes) og går hen og sætter sig på en stol med en bog/måske falder han i søvn.}

\scene Eksaminant ser forvirret ud -- og derefter skuffet.

\says{AV}Jeg er fortabt!
Musik: Skuffet, Weltschmertz – die Ende.

\scene DNA-feen kommer ind med en fabelagtig entré.

\textbf{DNA-sangen spilles.}

\scene Til sidst i sangen vågner eksamensvagten og opdager, at DNA-feen hjælper eksaminanten og afbryder sangen.\\
Evt. lyden af en plade, der får flået nålen af eller måske bare bandet, der stopper brat, hvorefter musikken toner over i dramatisk stumfilmsmusik.

Eksamensvagten er vred og fægter vildt med armene.

\says{AV}”HOV! Det må man ikke! Denne eksamen er uden hjælpemidler!”

\act{Eksamensvagten jager DNA-feen ud af scenen og eksaminanten sidder hjælpeløs tilbage på scenen.}\\
Lys ned. 



%Eksaminant spørger nu eksamensvagt ved gestikuleren.
%
%\says{AV}Goddag min gode mand, der må være sket en fejl. Lac-operonen er så sandelig ikke en del af pensum i Almen Molekylærbiologi.
%Musik: Samme.
%
%Eksamensvagt nikker, kigger, nikker igen forstående.
%
%\says{AV}Åhr hold så din kæft, jeg er jo bare vagt.
%Musik: Dum dum dumm!
%
%Eksaminant ser forvirret ud -- og så derefter skuffet.
%
%\says{AV}Jeg er fortabt!.
%Musik: Skuffet, Weltschmertz -- die Ende.
%
%Eksaminant bliver nu vred på eksamensvagten og på hele verden.
%
%\says{AV}Det kan de fandme ikke gøre mod mig! For dårlig!
%Musik: -Ingen
%
%Eksaminant kigger meget slesk og snu rundt.
%
%\says{AV}Så bliver jeg jo nødt til at finde andre metoder!
%Musik: Slesk, hekse-agtig musik?? Snyderier, udspekuleret.
%
%Eksaminant hiver i smug en bog op af sin taske og kigger i den om lac-operonen. 
%
%Eksamensvagt kommer og slår hårdt ned.
%
%\says{AV}HOV! Det må man ikke! Denne eksamen er uden hjælpemidler!
%Musik: Opgør.
%
%Eksaminant falder ned over bordet i fortvivlelse. Så opdager han dog en lille ting -- at der er noget kan svare på.
%
%\says{AV}Hov... Det spørgsmål handler om indianer-lort, det kan jeg jo godt svare på!
%Musik: Positiv stemningsmusik
%
%Eksaminant kigger glad op og tæller på fingrene.
%
%\says{AV}Det er 50\% af opgaven -- 02!!!
%Musik: Glad, stillleben, morgenstund.
%
%Eksaminant rejser sig op i glædesrus for han har bestået molbio.
%
%Lys ned.


\end{sketch}
\end{document}