\documentclass[a4paper,12pt]{article}

\usepackage{revy}
\usepackage[utf8]{inputenc}
\usepackage[T1]{fontenc}
\usepackage[danish]{babel}
\pdfoptionpdfminorversion 4

\revyname{Biorevy}
\revyyear{2013}
% HUSK AT OPDATERE VERSIONSNUMMER
% UNDLAD AT SKRIVE I TEMPLATE.TEX - KOPIÉR OG OMDØB I STEDET FOR
\version{2}
\eta{$X$ minutter}
\status{Færdig}

\title{Studenterpræsten - Ep. II}
\author{Biorevy}

\begin{document}
\maketitle


%\begin{texxers}
%	\texxer{Navn}[email@exmpl.com]
%\end{texxers}


\begin{roles}
	\role{SP}[Markus D] Studenterpræsten Prædbjørn
\role{Revyst}[Revyst] Revyst, der trækker Prædbjørn af scene
\role{Revyst}[Revyst] Revyst, der trækker Prædbjørn af scene
	\role{INS}[Thomas BT] Instruktør
\end{roles}


\begin{props}
	\prop{Powerpoint}[Markus D]PowerPoint-slide
	\prop{Prædbjørns tøj}[Markus D]Præstekrave og baggy jeans.
\end{props}

\begin{sketch}



\act{Studenterpræsten har nu lavet sin første introduktion der desværre ikke endte så godt. Nu forsøger han igen at fortsætte sit slideshow}

\says{SP} Hej igen! Jo, altså jeg fik vist ikke heeelt fortalt færdig, hvad jeg - studenterpræsten, kan gøre for jer! Det er mig!

\says{SP} For det er VIRKELIG vigtigt at have en Studenterpræst ved hånden i mange situationer. For Gud er der! Selv når sindet er tungest, tager Gud dig med til ham. Altså, ikke, øh. Nej... 

\says{A/V} A/V slide: Begravelse

\says{SP} For alting har jo en ende. Der er intet smukkere end når en død person ligger der. Det gør mig rigtigt glad. Altså, ikke som i haha-glad, men mennesker dør jo faktisk tit! Og der kan jeg hjælpe med døden. Ehm, ikke selve døden men altså -

\says{SP} I har nok en gammel mormor? Så bare kom til Prædbjørn, jeg ordner hende! Okay altså, ikke at jeg slår hende ihjel haha. Men det sidste hun kommer til at se, er mig. Altså, på den gode måde. Altså, jeg vil bare ordne hende - men ikke på den måde, men til at hun kan dø altså men ikke på den måde, men så jeg kan sende hende til Gud med mine egne hænder! Altså nej...

\act{To biorevyster kommer og trækker SP af scenen}

\end{sketch}
\end{document}