\documentclass[a4paper,12pt]{article}

\usepackage{revy}
\usepackage[utf8]{inputenc}
\usepackage[T1]{fontenc}
\usepackage[danish]{babel}
\pdfoptionpdfminorversion 4

\revyname{Biorevy}
\revyyear{2013}
% HUSK AT OPDATERE VERSIONSNUMMER
% UNDLAD AT SKRIVE I TEMPLATE.TEX - KOPIÉR OG OMDØB I STEDET FOR
\version{0}
\eta{$X$ minutter}
\status{Færdig, skal rettes lidt til}

\title{Snapselauget}
\author{Stine, Markus Back, Zabrine, Valbjørn og Helene}

\begin{document}
\maketitle


%\begin{texxers}
%	\texxer{Navn}[email@exmpl.com]
%\end{texxers}


\begin{roles}
	\role{S1}[Isabel HR] Snapsemand M/K 1
	\role{S2}[Alina J] Snapsemand M/K 2
	\role{B}[Casper G] Butler
	\role{INS}[Andreas N] Instruktør
\end{roles}


\begin{props}
	\prop{Brændeneældesnaps}[Hvem skaffer?] Brændeneældesnaps
	\prop{Egetræssnaps}[Hvem skaffer?] Egetræssnaps
	\prop{Gnaversnaps}[Hvem skaffer?] Gnaversnaps, brunlig
	\prop{Guld snaps}[Hvem skaffer?] Guld snaps
	\prop{Pandasnaps}[Hvem skaffer?] Pandasnaps
\end{props}

\begin{sketch}

\scene Der sidder to mænd på to stole ved siden af hinanden med et par glas, evt et par flasker med diverse bær, grene, blyanter, whatever. Der kan være tomme stole stående i en halvkreds. 

\says{S1}[højtideligt]: Velkommen til Det [Biologiske](find på et andet navn) Snapselaugs årlige snapsekonkurrence! Vi skal igen i år have kåret den fineste snaps! Jeg er glad for det store fremmøde.. kigger rundt, finder ud af de kun er to. Øh.. Du, Klaus hvor er de andre henne?

\says{S2}Altså... vi har haft lidt medlemsfrafald... 

\says{S1}Medlemsfrafald?

\says{S2}Jaaa, altså, Hans-Christian havde jo et mindre hjemmebrænderi, som desvære udviklede sig til en hmmm større hjemmebrand..
...Og altså, Göran forsvandt jo under hans ugentlige helsingør besøg.

\act{Lægger stille ud, skåler og smager efter hver}

\says{S1}Årh, (røvsnaps). Jamen, jeg vil gerne lægge ud - jeg har taget en snaps med som er brygget på brændenælde fra Himalaya.

\act{Smager/skåler}

\says{S2}Arrh, er det ikke lidt kedeligt? 

\act{hiver en snaps frem fra under stolen. }

\says{S2}Jeg har medbragt en Egetræssnaps, som gennem generationer er blevet modnet i et levende egetræ. 

\says{S1}I et egetræ?

\says{S2}Ja altså, de sidste 10 generationer af min familie har dagligt slidt og, eller altså tjenerne har, slæbt rigelige mængder sprit og vandet, og nu kan man så tappe direkte fra barken den fineste egesnaps.
(Evt. billede på AV -- træ med taphane)

\act{de skåler, drikker. Anerkendelse. }

\says{S1}Naah, den er lidt tør, det er der sgu ikke så meget ved. (pun intended) 
\says{S1}Nu skal du se hvad jeg har her: Snaps lavet på små bær der har passeret igennem en mindre gnaver. 

\scene Skåler i en lidt brunlig snaps (den er lavet på lort -- det er sjovt)

\says{S2}kigger på snapsen, skærer ansigt den skulle måske have været filtreret en gang mere...

\says{S1}Hørt Hørt! og skål!

\scene Smager/skåler

\says{S2}Jaja, den er da ganske udmærket, meen...

\says{S2}næh nej, Guld snaps. Denne snaps er lavet på guldflager, frisk fra fordøjelseskanalen på en gepard! 

\scene Smager/skåler

\says{S1}Mmmm ja, man kan ligefrem smage pletterne ikke!

\says{S2}Mjaaaa! laver kattefagter - grr/miaaau.. 

\says{S1}Ja den var sku lækker, nu vi er i dyreriget må jeg lige indvie dig en eksklusiv klub ikke... Tuk, Flasken!!! Tuk henter en flaske. Pandasnaps du!

\says{S2}Panda?

\says{S1}Ja for fanden, der går intet mindre end tre pandaer på en flaske!

\scene De smager, skåler, snapsen tages vel imod. 

\says{S2}Har du overvejet at lave den på rød panda istedet?

\says{S1}Tro mig, jeg har allerede sat flere flasker over

\says{S2}Hahaha men min gode mand, du er jo nybegynder i krybskyttesnaps... Næææ her ser du min egen trebjørnssnaps! 

\says{S1}Sagde du trebørnssnaps?!? 

\says{S2}Nej nej, tre- BJØRNSsnaps: Pandabjørn, koalabjørn og selvfølgelig isbjørn. Se, dét er en snaps der smager af verdensmand! Pandasnaps i sig selv bliver mig hurtigt noget sort/hvidt. 

\says{S1}Det smager jo langt væk af globalisering. Smukt!

\scene Pause, stemningen bliver lidt trykket. 

\says{S1}Nu har vi efterhånden lavet snaps på det meste, har vi ikke?

\says{S2}Ja, det har vi vist. Det bliver sgu svært at overgå sidste års vinder, lavet på Ulands bistand og afrikanske konflikt-diamanter. Det var sgu fair trade!

\says{S1}Hvad skal vi så finde på til næste år?

\says{S2}Vi kunne lave snaps på større primater? \act{ser på Tuk der vasker gulv} Kunstpause -- øjenkontakt -- nikker

\scene \textbf{I kor: Butlersnaps!}

\scene Lys ned


\end{sketch}
\end{document}