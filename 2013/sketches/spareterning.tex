\documentclass[a4paper,12pt]{article}

\usepackage{revy}
\usepackage[utf8]{inputenc}
\usepackage[T1]{fontenc}
\usepackage[danish]{babel}
\pdfoptionpdfminorversion 4

\revyname{Biorevy}
\revyyear{2013}
% HUSK AT OPDATERE VERSIONSNUMMER
% UNDLAD AT SKRIVE I TEMPLATE.TEX - KOPIÉR OG OMDØB I STEDET FOR
\version{2}
\eta{$4$ minutter}
\status{Færdig}

\title{Spareterningerne}
\author{Kresten, Helene, Nathalia, Valdbjørn, Drag}

\begin{document}
\maketitle


%\begin{texxers}
%	\texxer{Navn}[email@exmpl.com]
%\end{texxers}


\begin{roles}
	\role{Dek}[Andreas N]Dekanen
	\role{RR}[Kresten S]Rektor Ralf
	\role{SP}[Marcus G]Speak
	\role{INS}[Søren PH] Instruktør
\end{roles}


\begin{props}
	\prop{Spareterningerne}[Andreas N] Spareterningerne -- alm. terninger
	\prop{Pult}[Kresten S]
\end{props}

\begin{sketch}


\textbf{Bemærkninger til Teknikken:}

\says{Dek} Tillykke med din genansættelse Ralf! Det har du fortjent! Hvordan fik du i øvrigt krejlet den så du fik udbetalt en fratrædelsesgodtgørelse selvom du blev genansat?

\says{RR} HAHA. Det var intet problem, jeg lod simpelthen være med at gøre noget som helst! Hvorfor skulle jeg dog pille ved den ordning?

\says{Dek} Ja, hvorfor dog spare et sted, hvor det ikke går udover underviserne eller de studerende?

\act{Stilhed i et stykke tid }

\says{Dek} Men kære Ralf, må jeg ikke spørge dig om noget. Hvordan kan du være så effektiv i dine besparelses-beslutninger? De sidder jo lige i skabet hvert år!

\says{RR} Kan du holde på en lille hemmelighed? Tilbage i 2008 udviklede jeg et nyttigt redskab for alle mennesker i en chefstilling -- SPARE-TERNINGERNE!

\says{Dek} Neejjj! \act{Meget begejstret}

\says{RR} Disse dyrebare skatte har hjulpet mig nemt udenom mange hovedpiner. Ser du; Alle terningerne har et ord på sig, og når jeg så slår med dem, så ender de med at give en instruks. Dette kunne være noget harmløst som ``Fyr. Matematik. Ansatte. Kontor. Én.''

\says{Dek} Det er godt nok smart. Hvornår brugte du dem første gang på stor skala?

\says{RR} Aåhhh ja. Salig første gang. Jeg mindes året 2010, hvor en meddelelse udløb fra Christiansborg om at der skulle spares\ldots

\act{Ralf mindes saligt hans første gang med terningerne}

\says{Dek}[om en lille dreng i en slikbutik] Hvad sagde de så?

\says{RR}Fyr! Biologi! Ansatte! Planter! Alle! JAAAA.

\says{Dek} Aaah, så det var sådan du fik den genistreg.

\says{RR} Ja, de eneste planter jeg vil se på er sgu græsset på golfbanen.

\says{Dek} Men man behøver vel ikke altid følge terningernes råd?

\says{RR} JO FOR HELVEDE! SPARE-Terninger er en hellig instans, og dem må der ikke pilles ved.

\says{Dek}[vil se lidt magi] Kunne du ikke give en demonstration af sparer-terningers kræfter her nu, foran mig?

\says{RR} Vil du virkelig udfordrer SPARETERNINGERNES Hellige kraft?

\says{Dek} Mmmm\ldots \act{Nikker}

\says{RR} Meget vel, nu hvor det også snart er tid til at ligge budgetter igen, gør det vel ikke noget vi tyvstarter!

\act{Slår med terninger}

\says{RR} Bortskaf! Hele! Studie! Miljø! Puljen!

\act{Dekanen og Ralf udbryder begge højt "Jaaaa", og omfavner hinanden. Ralf samler terningerne op, og gør sig klar til at putte dem i lommen}

\says{Dek} HAHA! Så er den ged vist barberet! 

\says{Dek} Men ehhh, Ralf sig mig -- skal vi ikke prøve terningerne igen?

\act{Stilhed. De to herrer kigger på hinanden.}

\says{RR}Jooo\ldots Det kan vel ikke skade. Vi kunne også godt bruge lidt mere luft i budgettet.

\says{RR} GODT! Det gør vi! KAST!

\act{Dekanen kaster med terningerne.}

\says{Dek} Afskaf! Pension! Midler! Ledelse! Nu! \act{med hjertet helt oppe i halsen}

\says{RR} DIN IDIOT! SE NU HVAD DU HAR GJORT!

\says{Dek} Tag det nu ikke så tungt; Det der ser vi da bare bort fra..

\says{RR} TERNINGERNE -- SKAL -- ADLYDES!

\act{Begge herrer er paniske. Dog finder Dekanen på noget.}

\says{Dek} Nej nej! Jeg ved hvad vi gør! Vi spørger øh øh\ldots

\act{Hiver en banan op.}

\says{Dek} Beslutnings-Bananen$^{TM}$ !

\act{Ralf er meget skeptisk over for beslutnings-bananen.}

\says{RR} Det er jo bare en banan for helvede!

\says{Dek} Nej nej! Det er Beslutnings-bananen! Lad os høre hvad den siger!

\scene Der er en pause, hvor Ralf og Dekanen venter i spænding på hvad BeslutningsBananen$^{TM}$ siger.

\says{Banan (speak)} Terningerne skal altid adlydes!

\says{Begge i kor} NEEEEEJ!
\end{sketch}
\end{document}